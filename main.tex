\makeatletter
\newcommand\HUGE{\@setfontsize\Huge{20}{30}}
\makeatother    
 
\newcommand{\Titolo}{Computational Mathematics for Learning and Data Analysis}

\newcommand{\Gruppo}{Federico Finocchio\\\texttt{f.finocchio@studenti.unipi.it}}
\newcommand{\norm}[1]{\left\lVert #1 \right\rVert_2}

\documentclass[a4paper, oneside, openany]{article}
\usepackage[backend=biber, sorting=none]{biblatex}
\addbibresource{bibliography.bib}
\usepackage{todonotes}
\usepackage{titling}
\usepackage{graphbox}
\usepackage{amsmath , amssymb , amsthm}
\usepackage{comment} 
% permette di modificare i margini
\usepackage[top=3.1cm, bottom=3.1cm, left=2.2cm, right=2.2cm]{geometry}

\usepackage{lastpage} %info sul # dell'ultima pagina del documento
\usepackage{fancyhdr} %per modificare dimensioni,margini, intestazioni e righe a piè di pagina
\fancypagestyle{plain}{
  % cancella tutti i campi di intestazione e piè di pagina
  \fancyhf{}
  
  \rfoot{Page \thepage{} of \pageref{LastPage}} %es: pag: 4 di 10

  %linea orizzontale alle posizioni top e bottom della pagina
  \renewcommand{\headrulewidth}{0.2	pt}  
  \renewcommand{\footrulewidth}{0.2pt}
}
\pagestyle{plain} 

%\usepackage{calc} %introduce la notazione infissa per le op. aritmetiche interne a LaTeX

\usepackage[utf8]{inputenc}
\usepackage[T1]{fontenc}
\usepackage[english]{babel} %il documento è in italiano

\usepackage{graphicx}       %permette di inserire delle immagini
\usepackage{caption}        %numerazione figure e loro descrizione testuale
\usepackage{subcaption}     %sottofigure numerabili
\usepackage{float}  %permette di inserire un # qualsiasi di figure fluttuanti
\usepackage{xcolor}
\usepackage{rotating} %permette di ruotare le immagini

%package utili per la math mode ( $ ... $ o \[ ...  )
\usepackage{amsmath}
\usepackage{amssymb}
\usepackage{amsfonts}
\usepackage{amsthm}
\usepackage{mathtools}

% package utili per tabelle(\thead in particolare)
\usepackage{array, booktabs, caption}
\usepackage{makecell}
\renewcommand\theadfont{\bfseries}
\usepackage{boldline}

\usepackage{listings} %permette di inserire degli spezzoni di codice

\usepackage{tikz} %disegno di immagini vettoriali a schermo. Utile per grafi
\usetikzlibrary{arrows.meta}
\usetikzlibrary{graphs}
\usetikzlibrary{arrows}
%\usepackage{tikz-uml} %serve per disgnare l'UML, fantastica guida:
%https://perso.ensta-paristech.fr/~kielbasi/tikzuml/var/files/doc/tikzumlmanual.pdf
%download package: http://perso.ensta-paristech.fr/~kielbasi/tikzuml/
\usepackage{fix-cm}    
%package per le tabelle
\usepackage{booktabs} %permette di poter usare delle liste nelle tabelle
\usepackage{tabularx} 
\usepackage{longtable} %una tabella può continuare su più pagine
\usepackage{multirow} %utile per visualizzare una cella su più righe
%\usepackage{multicolumn} %cella su più colonne
%\usepackage[table]{xcolor} %rende disponibile l'utilizzo di un colore per lo sfondo
                        %delle celle di una tabella

%crea una cella per le tabelle in grado di andare a capo con \newline
%https://tex.stackexchange.com/questions/12703/how-to-create-fixed-width-table-columns-with-text-raggedright-centered-raggedlef
\usepackage{array}
\newcolumntype{L}[1]{>{\raggedright\let\newline\\\arraybackslash\hspace{0pt}}m{#1}}
\newcolumntype{C}[1]{>{\centering\let\newline\\\arraybackslash\hspace{0pt}}m{#1}}
\newcolumntype{R}[1]{>{\raggedleft\let\newline\\\arraybackslash\hspace{0pt}}m{#1}}


%indice con i puntini
\usepackage{tocloft}
\renewcommand\cftsecleader{\cftdotfill{\cftdotsep}}

%http://ctan.mirror.garr.it/mirrors/CTAN/macros/latex/contrib/appendix/appendix.pdf
\usepackage{appendix} %aggiunge dei comandi per l'appendice
\usepackage{parskip} %aiuta LaTeX a trovare il miglior stile per i page break
\setcounter{secnumdepth}{5} % numera i sottoparagrafi
\setcounter{tocdepth}{5} %aggiunge all'indice i sottoparagrafi
%\usepackage{titlesec} %\begin{paragraph} si può usare come subsubsubsection!


\usepackage{breakurl}%\url{...} può continare alla linea successiva. (si può andare a capo)

\definecolor{Maroon}{cmyk}{0, 0.87, 0.68, 0.32}
\usepackage[colorlinks=true]{hyperref}
\hypersetup{
    colorlinks=true,
    citecolor=black,
    filecolor=black,
    linkcolor=black, % colore dei link interni
    urlcolor=Maroon  % colore dei link interniesterni
}

%impostazioni per il codice che deve finire dentro a
%\begin{lstlisting}

\definecolor{listinggray}{gray}{0.9}
\definecolor{lbcolor}{rgb}{0.9,0.9,0.9}
\lstset{
backgroundcolor=\color{lbcolor},
    tabsize=4,    
%   rulecolor=,
    language=[GNU]C++,
    basicstyle=\scriptsize,
    upquote=true,
    aboveskip={1.5\baselineskip},
    columns=fixed,
    showstringspaces=false,
    extendedchars=true,
    inputencoding=utf8,
    breaklines=true,
    prebreak = \raisebox{0ex}[0ex][0ex]{\ensuremath{\hookleftarrow}},
    frame=single,
    numbers=left,
    showtabs=false,
    showspaces=false,
    showstringspaces=false,
    identifierstyle=\ttfamily,
    keywordstyle=\color[rgb]{0,0,1},
    commentstyle=\color[rgb]{0.026,0.112,0.095},
    stringstyle=\color[rgb]{0.627,0.126,0.941},
    numberstyle=\color[rgb]{0.205, 0.142, 0.73},
%        \lstdefinestyle{C++}{language=C++,style=numbers}’.
}
\lstset{
  backgroundcolor=\color{lbcolor},
  tabsize=4,
  language=C++,
  captionpos=b,
  tabsize=3,
  frame=lines,
  numbers=left,
  numberstyle=\tiny,
  numbersep=5pt,
  breaklines=true,
  showstringspaces=false,
  basicstyle=\footnotesize,
  identifierstyle=\color{magenta},
  keywordstyle=\color[rgb]{0,0,1},
  commentstyle=\color{orange},
  stringstyle=\color{red}
}

\usepackage{algorithm}% http://ctan.org/pkg/algorithms
\usepackage{algpseudocode}% http://ctan.org/pkg/algorithmicx

 \newgeometry{top=4cm}

\begin{document}

\begin{titlepage}
	\begin{center}
		
		\vspace{1cm}
	
		\begin{HUGE}
		
			\Titolo{} \\
		\end{HUGE}
		
		\vspace{13pt}  
		
		\begin{large}
		\Gruppo{}\ \\	
		\end{large}
		
		\vspace{10pt}
		    
		\begin{large}
			A.Y. 2020/2021
		\end{large}  
		
	\end{center}
	\vspace{1cm}
\begin{abstract}
\textit{Assigned project: ML project 6}\newline
(M1) is a neural network with topology of your choice, but mandatory piecewise-linear activation function (of your choice); any regularization is allowed.\newline
(M2) is a standard L\_2 linear regression (min least squares).\newline
(A1) is a standard momentum descent approach applied to (M1).\newline
(A2) is an algorithm of the class of deflected subgradient methods applied to (M1).\newline
(A3) is a basic version of the direct linear least squares solver of your choice (normal equations, QR, or SVD) applied to (M2).
\end{abstract}
\end{titlepage}

\restoregeometry

\section{Introduction}
This report is about the project assigned for the course of Computational Mathematics for Learning and Data Analysis. All the work in this project is the result of the knowledge gathered from the courses of \textbf{ML} and \textbf{CM}. The report contents that are not directly work of the authors is referenced and, as requested, we point to the references down to chapter and number of page (when necessary).

We start by giving a short description of the problem at hand and the methods used to solve it, including all the mathematical derivation needed to adapt the chosen methods to the problem. Next, we give a brief recap of the expected results for the experiments, properties of the problem that suits our methods and details about the solvability of our problem with the used methods. In the end, we show the achieved results, comparing them with the expected one describing which are the factors that determined a difference in the results.

The models to be implemented are:
\begin{itemize}
    \item \textbf{M1}: a neural network, \textit{ANN} in the following, with piecewise-linear activation function, with possible regularization;
    \item \textbf{M2}: standard L\_2 linear regression
\end{itemize}
The methods to be applied to the models are:
\begin{itemize}
    \item \textbf{A1}: standard momentum descent approach applied to \textbf{M1};
    \item \textbf{A2}: deflected subgradient methods applied to \textbf{M1};
    \item \textbf{A3}: basic version of one of the direct linear least squares solvers (i.e. normal equations, QR, SVD) applied to \textbf{M2}.
\end{itemize}
In the following we describe the main implementation choices and introduce some of the notation used in the rest of the report. The detailed description of the implemented methods is given in the related sections of this document.
\section{Project structure and notation}
The main aim of this project was to build two application models and apply different optimization methods to the implemented models.\newline
The models to be implemented are:
\begin{itemize}
    \item \textbf{M1}: a neural network with piecewise-linear activation function, with possible regularization;
    \item \textbf{M2}: standard L\_2 linear regression
\end{itemize}
The methods to be applied to the models are:
\begin{itemize}
    \item \textbf{A1}: standard momentum descent approach applied to \textbf{M1};
    \item \textbf{A2}: deflected subgradient methods applied to \textbf{M1};
    \item \textbf{A3}: basic version of one of the direct linear least squares solvers (i.e. normal equations, QR, SVD) applied to \textbf{M2}.
\end{itemize}
In the following we will describe what are the main implementation choices and introduce some of the notation used in the rest of the report.\newline
As requested, the project requires to implement from scratch an \textit{Artificial Neural Network}, \textit{ANN} in the following and as referred in \cite{MLmitchell}, and a direct solver for the linear least square problem. The detailed description of the implemented methods will be given in the related sections of this document.
\section{Artificial Neural Network}
The notation used to refer The implemented \textit{ANN} can be seen as a \textit{fully connected multilayer Perceptron} as referred in the literature and as shown in \cite{MLmitchell}. An \textit{ANN} is composed by an interconnection of units that can represented as the composition of two functions that will determine the real-valued output of \todo{magari descrivere che cosa è un perceptron}the unit. The two functions will be referred as the \textit{network function} and the \textit{activation function}, where the former computes the scalar product of the input vector with the weight vector of the current unit, the latter is the function that will directly determine the output of the current unit. As we will see, the choice of the activation function is particularly important and is preferred to be a nonlinear combination of its inputs, maintaining the property of being differentiable.

The implemented \textit{ANN} will be structured with multiple layers, each layer will have all the units fully connected with the adjacent layers and, as convention, we will refer to the first layer as \textit{input layer} and to the last layer as \textit{output layer}. The others will be referred as \textit{hidden layers}. Another important aspect when implementing an \textit{ANN} is the choice of the number of units. Later in this report will be shown how the exact number of units in each layer will be chosen, but we can already describe the structure of the input and the output layer. The former will contain a number of units that is the same as the number of features contained in the data that will be fed up to the \textit{ANN}, instead the latter will contain one binary output units for classification tasks and one real-output unit for regression tasks.\newline

In the following sections we will describe in more details which are the main aspects of the implemented \textit{ANN} like the network structure, the functions used to compute the output of each unit and the algorithm used to let the network learn the task at hand.

\subsection{Network structure}
As already pointed out in the previous section, the network structure will be composed of multiple layers. The number of layers and consequently, the number of units per layer will be determined by an empirical approach that will target the minimization of the empirical error on the tested data. This will be explained in more detail in the testing section.\newline

The input and the output layer will be of fixed dimensions, as the number of units for the former will be determined by the size of the input data and the dimension of the latter will be determined by the task to be completed. As we will see, the number of units in the output layer will be always of one, what will change instead is the type of the units, because for classification tasks we will use a binary unit, instead for a regression task we will use a real-output unit.

\subsection{Activation function}
The choice of the activation function is a crucial step for the construction of the \textit{ANN}. This function will directly determine what is the output of each of the units in the network, depending on the result of the scalar product of the unit input and the unit weights vectors. The activation function, to be useful, needs to have determined properties, like differentiability, \textbf{AGGIUNGERE ALTRE PROPRIETà CHE LA RETE DEVE AVERE}.\newline

The mainly used ones, which will be used also for the purposes of this project, are:
\begin{itemize}
    \item Sigmoid
    \item TanH
\end{itemize}

\subsection{Loss function}
The loss function is used to estimate the error at the output of the network given the vector of the expected values for the data that are fed up into the network. The loss function is also the function to be minimized as the main aim for the learning algorithm. The error computed via the loss function is usually used as a termination condition for the learning algorithm. This is done via various algorithm, and in particular for our case via subgradient methods and standard momentum descent approach.\newline

In our case we decided to use as a function to measure the error in the prediction the \textit{MSE} that represents the sum, over all the available data, of the squared differences between the predicted value and the actual one.\newline

\subsubsection{Derivation of Loss function}
As a preliminary step for the backpropagation algorithm, we illustrate here how the gradient of the loss function is computed. The result is then used to update the value of the weight vector for each unit. In the following the derivation of the loss function.\newline\newline
\textbf{QUI INSERIRE DERIVAZIONE LOSS FUNCTION}

\subsection{Backpropagation algorithm}
The backpropagation algorithm \cite{haykin_neural_2009} is used to learn weight vectors for a multilayer network, given the network inputs and the fixed network structure (i.e. units and interconnections). This algorithm employs a gradient descent approach to attempt to minimize the squared error between the network output values $\textbf{y}$ and the target values $\hat{\textbf{y}}$ associated to these outputs. This algorithm is described in \cite{MLmitchell} and is composed by two main parts:
\begin{itemize}
    \item \textbf{Forward phase}: data traverse the network from the input units to the output units, in such a way the result of the network with the given input and the current weight vector value can be compared to the expected output to estimate the error;
    \item \textbf{backward phase}: the gradient of the loss function is used to update the weight vectors of each layer in a way that the next step will have a smaller value for the error function.
\end{itemize}
The main aim for the backpropagation algorithm is to minimize the error function via automatic fine-tuning of the weight vector. This minimization can be achieved in different ways utilizing the gradient of the loss function computed with the results obtained with the forward phase.\newline
\cite{MLmitchell}

\section{Least Square}
\label{sec:ls}
The \textit{Least Square problem} is described in \parencite[Lecture 11]{Bau} and \parencite[Chap. 3]{elden} as the problem of finding a solution of an overdetermined system of equations $Ax = b$ by finding a vector $x$ that minimizes the 2-norm of the residual vector defined as $r = b - Ax$.

The \textit{Least Square problem}, given $A\in\mathbb{R}^{m\times n},\ m\geq n,\ b\in\mathbb{R}^m$, has the following form:
\begin{equation}
    \label{eq:ls}
    \text{find}\ x\in\mathbb{R}^n\ \text{that solves the minimization problem}\ \min_{x}\norm{b - Ax}\ \text{.}
\end{equation}
We describe in section \S\ref{sec:qr}, related to the implemented direct solver, that this kind of problems have a unique solution if the matrix $A$ has \textit{full column rank}.
\section{Methods}
This section will give detailed information about the required methods that will be implemented and applied to the optimization problems described in \S\ref{sec:ann} and \S\ref{sec:ls}. The main methods to be implemented are:
\begin{itemize}
    \item Standard momentum descent approach applied to the \textit{ANN};
    \item Deflected subgradient method applied to the \textit{ANN};
    \item Direct linear least square solver applied to the \textit{Least Square problem}.
\end{itemize}

\subsection{Direct solver for Linear Least Square}
\label{sec:qr}
We have chosen to implement the direct solver via \textit{QR factorization}. This section will give a detailed description of all the properties needed for a \textit{least square problem} to be solved via this method and all the expected results for this kind of implementation. In a successive section we plan to insert the comparison between the theoretical results shown in this section and the actual result obtained in testing the implemented algorithm.
\subsubsection{QR factorization}
\label{subsec:qr}
As described in \parencite[Chap. 5]{elden}, \textit{QR decomposition} (or factorization) is a factorization of a matrix $A$ in a product of an orthogonal matrix and a triangular matrix obtained via successive orthogonal transformations.
\newtheorem{thm}{Theorem}[section]
\newtheorem{lemma}[thm]{Lemma}
\begin{thm}
\label{thm:qr}
Any matrix $A\in \mathbb{R}^{m\times n},\ m\geq n$, can be transformed to upper triangular form by an orthogonal matrix. The transformation is equivalent to a decomposition
\begin{align*}
    A = Q\begin{bmatrix}R \\ 0\end{bmatrix},
\end{align*} where $Q\in \mathbb{R}^{n\times n}$ is upper triangular. If the columns of $A$ are linearly independent, then $R$ is nonsingular.
\end{thm}
\todo[inline]{proof of this theorem is on \cite{elden} page 59}
If we partition $Q = (Q_1\ Q_2)$ where $Q_1\in \mathbb{R}^{m\times n}$, noting that in the multiplication $Q_2$ is multiplied by zero, we can write:
\begin{equation}
\label{eq:thinqr}
    A = \begin{bmatrix}Q_1 Q_2\end{bmatrix}\begin{bmatrix}R \\ 0\end{bmatrix} = Q_1R
\end{equation}
Where \hyperref[eq:thinqr]{equation (\ref{eq:thinqr})} refers to the \textbf{thin QR factorization}. This form of the \textit{QR factorization} will be the one used from now on to solve the \textit{linear least square problem} due to its efficiency in space and time.
\todo[inline]{trovare giustificazione per questa affermazione. Magari mostrare quali sono i vantaggi rispetto al QR normale.}

\begin{lemma}
For every $v\in \mathbb{R}^m$, the matrix $H = I - \frac{2}{v^Tv}vv^T = I - \frac{2}{\norm{v}^2}vv^T
 = I - 2uu^T,\ (where\ u=\frac{1}{\norm{v}}v\text{ has norm 1})$ is orthogonal.
\end{lemma}
\begin{lemma}
Let $x, y$ be two vectors such that $\norm{x} = \norm{y}$. If one chooses $v=x-y$, then $H = I - \frac{2}{v^Tv}vv^T$ is such that $Hx = y$.
\end{lemma}
By choosing $y = \norm{x}e_1 = \begin{bmatrix}\norm{x}\\0\\ \vdots \\0\end{bmatrix}$ we can build a procedure to find the householder vector $\textbf{u}$ of a generic vector $\textbf{x}$. The pseudocode for this procedure is shown in Algorithm \ref{alg:hh}.
\begin{algorithm}[H]
	\caption{Householder vector}
	\label{alg:hh}
	\begin{algorithmic}[1]
		\Function{householder\_vector}{x}
		\State $\mathbf{s} \leftarrow norm(x)$
		\State $\mathbf{v} \leftarrow x$
		\State $\mathbf{v}[1] \leftarrow \mathbf{v}[1] - s$
		\State $\mathbf{u} \leftarrow \mathbf{v} / norm(\mathbf{v})$
	\end{algorithmic}
\end{algorithm}
\todo[inline]{aggiungere riferimento a capitolo \cite{elden} in cui si definiscono householder transformations}
\todo[inline]{magari aggiungere pseudocodice con cui si calcola householder vector}
Now we illustrate the method used to compute this factorization through the \textit{Householder transformation}. By a sequence of orthogonal transformation we can transform any matrix $A\in \mathbb{R}^{m\times n}, m\geq n$,
\begin{align*}
    A \to Q^TA = \begin{bmatrix}R \\ 0\end{bmatrix}, R \in \mathbb{R}^{n\times n}
\end{align*}
where $R$ is upper triangular and $Q\in \mathbb{R}^{m\times m}$ is orthogonal. As shown in \parencite[Chap. 5.1]{elden} we can illustrate the procedure using a smaller matrix $A\in \mathbb{R}^{5\times 4}$. Basically the algorithm proceeds by zeroing the elements under the main diagonal, where at each step $i$ the elements below the element $a_{i,i}$ are zeroed by left-multiplying the current matrix $A_i$ to a matrix $H_{i+1}$.

In the first step we zero the elements below the main diagonal in the first column:
\begin{align*}
    H_1A = H_1\begin{pmatrix}\text{x} & \text{x} & \text{x} & \text{x} \\ \text{x} & \text{x} & \text{x} & \text{x} \\ \text{x} & \text{x} & \text{x} & \text{x} \\ \text{x} & \text{x} & \text{x} & \text{x} \\ \text{x} & \text{x} & \text{x} & \text{x}\end{pmatrix} = \begin{pmatrix}\text{+} & \text{+} & \text{+} & \text{+} \\ 0 & \text{+} & \text{+} & \text{+} \\ 0 & \text{+} & \text{+} & \text{+} \\ 0 & \text{+} & \text{+} & \text{+} \\ 0 & \text{+} & \text{+} & \text{+}\end{pmatrix} = A_1,
\end{align*}
where \textbf{+} denotes an element that has changed in the transformation. The orthogonal matrix $H_1$ can be taken equal to a \textit{Householder transformation}. In the second step we use an embedded \textit{Householder transformation} to zero the elements below the diagonal of the second column of matrix $A_1$:
\begin{align*}
    H_2A_1 = H_2\begin{pmatrix}\text{x} & \text{x} & \text{x} & \text{x} \\ 0 & \text{x} & \text{x} & \text{x} \\ 0 & \text{x} & \text{x} & \text{x} \\ 0 & \text{x} & \text{x} & \text{x} \\ 0 & \text{x} & \text{x} & \text{x}\end{pmatrix} = \begin{pmatrix}\text{x} & \text{x} & \text{x} & \text{x} \\ 0 & \text{+} & \text{+} & \text{+} \\ 0 & 0 & \text{+} & \text{+} \\ 0 & 0 & \text{+} & \text{+} \\ 0 & 0 & \text{+} & \text{+}\end{pmatrix} = A_2
\end{align*}
And so on, after the fourth step we have computed the upper triangular matrix $R$. The sequence of transformations is summarized as:
\begin{align*}
    Q^TA = \begin{bmatrix}R \\ 0\end{bmatrix},\ Q^T=H_4H_3H_2H_1.
\end{align*}
Assuming $A\in \mathbb{R}^{m\times n}$ the matrices $H_i$ have the following structure:
\begin{flalign*}
    & H_1 = I - 2u_1u_1^T,\ u_1\in \mathbb{R}^m \\
    & H_2 = \begin{pmatrix}1 & 0 \\ 0 & P_2\end{pmatrix},\ P_2 = I - 2u_2u_2^T,\ u_2\in \mathbb{R}^{m-1} \\
    & H_3 = \begin{pmatrix}1 & 0 & 0 \\ 0 & 1 & 0 \\ 0 & 0 & P_3\end{pmatrix},\ P_3 = I - 2u_3u_3^T,\ u_3\in \mathbb{R}^{m-2}
\end{flalign*}
Noting that vector $u_i$, obtained with the procedure defined in Algorithm \ref{alg:hh}, becomes shorter at each step, we include \textit{Householder transformations} of increasingly smaller dimensions in identity matrices.

\subsubsection{Solving Least Square via QR factorization}
In this section we show how \textit{QR factorization}, shown in \S\ref{subsec:qr}, can be used to solve the \textit{linear least square problem} defined in equation (\ref{eq:ls}).\newline
In the following we use the fact that the Euclidean vector norm is invariant under orthogonal transformations, i.e. $\norm{Qy} = \norm{y}$.

\begin{thm}
\label{thm:ls_qr}
Let the matrix $A\in \mathbb{R}^{m\times n}$ have full column rank and thin QR decomposition $A = Q_1R$. Then the least squares problem $\min_{x} \norm{b - Ax}$ has the unique solution
\begin{align*}
    x = R^{-1}Q_1^Tb.
\end{align*}
\end{thm}
\begin{proof}
Introducing the QR decomposition of $A$ in the residual vector, we get
\begin{align*}
    \norm{r}^2 = \norm{b - Ax}^2 = \norm{b - Q\begin{bmatrix}R \\ 0\end{bmatrix}x}^2 = \norm{Q^Tb-Q^TQ\begin{bmatrix}R \\ 0\end{bmatrix}x)}^2 = \norm{Q^Tb - \begin{bmatrix}R \\ 0\end{bmatrix}x}^2
\end{align*}
Then we partition $Q = (Q_1\ Q_2)$, where $Q_1\in \mathbb{R}^{m\times n}$, so we can write
\begin{equation}
\label{eq:lsqr}
    \norm{r}^2 = \norm{\begin{bmatrix}Q_1^Tb \\ Q_2^Tb\end{bmatrix} - \begin{bmatrix}Rx \\ 0\end{bmatrix}}^2 = \norm{Q_1^Tb - Rx}^2 + \norm{Q_2^Tb}^2
\end{equation}
Under the assumption that the columns of $A$ are linearly independent and since the second term of equation (\ref{eq:lsqr}) is independent from vector $x$, we can solve $Rx = Q_1^Tb$ and minimize $\norm{r}^2$ by making the first term in \hyperref[eq:lsqr]{equation (\ref{eq:lsqr})} equal to zero.
\end{proof}
\printbibliography
\listoftodos

\end{document}
