\subsection{Direct solver for Linear Least Square}
\label{sec:qr}
We have chosen to implement the direct solver via \textit{QR factorization}. This section gives a description of all the properties needed for a \textit{least square problem} to be solved via this method and all the expected results for this kind of implementation. In a successive section we plan to insert the comparison between the theoretical results shown in this section and the actual result obtained in testing the implemented algorithm.
\subsubsection{QR factorization}
\label{subsec:qr}
As described in \parencite[Chap. 5]{elden}, \textit{QR decomposition} (or factorization) is a factorization of a matrix $A$ in a product of an orthogonal matrix and a triangular matrix obtained via successive orthogonal transformations.

\begin{thm}
\label{thm:qr}
Any matrix $A\in \mathbb{R}^{m\times n},\ m\geq n$, can be transformed to upper triangular form by an orthogonal matrix. The transformation is equivalent to a decomposition
\begin{align*}
    A = Q\begin{bmatrix}R \\ 0\end{bmatrix},
\end{align*} where $Q\in \mathbb{R}^{m\times m}$ is orthogonal and $R\in \mathbb{R}^{n\times n}$ is upper triangular. If the columns of $A$ are linearly independent, then $R$ is nonsingular.
\end{thm}
% \todo[inline]{proof of this theorem is on \cite{elden} page 59}
If we partition $Q = (Q_1\ Q_2)$ where $Q_1\in \mathbb{R}^{m\times n}$, noting that in the multiplication $Q_2$ is multiplied by zero, we can write:
\begin{equation}
\label{eq:thinqr}
    A = \begin{bmatrix}Q_1 & Q_2\end{bmatrix}\begin{bmatrix}R \\ 0\end{bmatrix} = Q_1R,
\end{equation}
where \hyperref[eq:thinqr]{equation (\ref{eq:thinqr})} refers to the \textbf{thin QR factorization}. This form of the \textit{QR factorization} is the one used from now on to solve the \textit{linear least square problem}.
% \todo[inline]{trovare giustificazione per questa affermazione. Magari mostrare quali sono i vantaggi rispetto al QR normale.}

To implement this factorization we make use of the \textit{Householder transformations} described in \parencite[Chap. 4.2.2]{elden}.

\begin{lemma}
For every $\mathbf{v}\in \mathbb{R}^m$, the matrix $\mathbf{H} = I - \frac{2}{v^Tv}vv^T = I - \frac{2}{\norm{v}^2}vv^T
 = I - 2uu^T,\ (where\ u=\frac{1}{\norm{v}}v\text{ has norm 1})$ is orthogonal. We call $\mathbf{H}$ an Householder transformation.
\end{lemma}
\begin{lemma}
Let $x, y$ be two vectors such that $\norm{x} = \norm{y}$. If one chooses $v=x-y$, then $H = I - \frac{2}{v^Tv}vv^T$ is such that $Hx = y$.
\end{lemma}
% \todo[inline]{cercare reference per questi lemma}
By choosing $y = \norm{x}e_1 = \begin{bmatrix}\norm{x}\\0\\ \vdots \\0\end{bmatrix}$ we can build a procedure to find the householder vector $\textbf{u}$ of a generic vector $\textbf{x}$. The pseudocode for this procedure is shown in Algorithm \ref{alg:hh}.
\begin{algorithm}[H]
	\caption{Householder vector}
	\label{alg:hh}
	\begin{algorithmic}[1]
		\State $\mathbf{s} \leftarrow norm(x)$
		\State $\mathbf{v} \leftarrow x$
		\State $\mathbf{v}[1] \leftarrow \mathbf{v}[1] - s$
		\State $\mathbf{u} \leftarrow \mathbf{v} / norm(\mathbf{v})$
	\end{algorithmic}
\end{algorithm}
Now we illustrate the method used to compute the QR factorization through the \textit{Householder transformation}. By a sequence of orthogonal transformation we can transform any matrix ${A\in \mathbb{R}^{m\times n}, m\geq n}$,
\begin{align*}
    A \to Q^TA = \begin{bmatrix}R \\ 0\end{bmatrix}, R \in \mathbb{R}^{n\times n}
\end{align*}
where $R$ is upper triangular and $Q\in \mathbb{R}^{m\times m}$ is orthogonal. As shown in \parencite[Chap. 5.1]{elden} we can illustrate the procedure using a smaller matrix $A\in \mathbb{R}^{5\times 4}$. Basically the algorithm proceeds by zeroing the elements under the main diagonal, where at each step $i$ the elements below the element $a_{i,i}$ are zeroed by left-multiplying the current matrix $A_i$ to a matrix $H_{i+1}$.

In the first step we zero the elements below the main diagonal in the first column:
\begin{align*}
    H_1A = H_1\begin{pmatrix}\text{x} & \text{x} & \text{x} & \text{x} \\ \text{x} & \text{x} & \text{x} & \text{x} \\ \text{x} & \text{x} & \text{x} & \text{x} \\ \text{x} & \text{x} & \text{x} & \text{x} \\ \text{x} & \text{x} & \text{x} & \text{x}\end{pmatrix} = \begin{pmatrix}\text{+} & \text{+} & \text{+} & \text{+} \\ 0 & \text{+} & \text{+} & \text{+} \\ 0 & \text{+} & \text{+} & \text{+} \\ 0 & \text{+} & \text{+} & \text{+} \\ 0 & \text{+} & \text{+} & \text{+}\end{pmatrix} = A_1,
\end{align*}
where \textbf{+} denotes an element that has changed in the transformation. The orthogonal matrix $H_1$ can be taken equal to a \textit{Householder transformation}. In the second step we use an embedded \textit{Householder transformation} to zero the elements below the diagonal of the second column of matrix $A_1$:
\begin{align*}
    H_2A_1 = H_2\begin{pmatrix}\text{x} & \text{x} & \text{x} & \text{x} \\ 0 & \text{x} & \text{x} & \text{x} \\ 0 & \text{x} & \text{x} & \text{x} \\ 0 & \text{x} & \text{x} & \text{x} \\ 0 & \text{x} & \text{x} & \text{x}\end{pmatrix} = \begin{pmatrix}\text{x} & \text{x} & \text{x} & \text{x} \\ 0 & \text{+} & \text{+} & \text{+} \\ 0 & 0 & \text{+} & \text{+} \\ 0 & 0 & \text{+} & \text{+} \\ 0 & 0 & \text{+} & \text{+}\end{pmatrix} = A_2
\end{align*}
And so on, after the fourth step we have computed the upper triangular matrix $R$. The sequence of transformations is summarized as:
\begin{align*}
    Q^TA = \begin{bmatrix}R \\ 0\end{bmatrix},\ Q^T=H_4H_3H_2H_1.
\end{align*}
Assuming $A\in \mathbb{R}^{m\times n}$ the matrices $H_i$ have the following structure:
\begin{flalign*}
    & H_1 = I - 2u_1u_1^T,\ u_1\in \mathbb{R}^m \\
    & H_2 = \begin{pmatrix}1 & 0 \\ 0 & P_2\end{pmatrix},\ P_2 = I - 2u_2u_2^T,\ u_2\in \mathbb{R}^{m-1} \\
    & H_3 = \begin{pmatrix}1 & 0 & 0 \\ 0 & 1 & 0 \\ 0 & 0 & P_3\end{pmatrix},\ P_3 = I - 2u_3u_3^T,\ u_3\in \mathbb{R}^{m-2}
\end{flalign*}
Thus vectors $\mathbf{u_i}$, obtained with the procedure defined in Algorithm \ref{alg:hh}, become shorter at each step and we embed \textit{Householder transformations} of increasingly smaller dimensions in identity matrices.
\subsubsection{Solving Least Square via QR factorization}
In this section we show how \textit{QR factorization}, shown in \S\ref{subsec:qr}, can be used to solve the \textit{linear least square problem} defined in equation (\ref{eq:ls}).\newline
In the following we use the fact that the Euclidean vector norm is invariant under orthogonal transformations, i.e. $\norm{Qy} = \norm{y}$.

\begin{thm}
\label{thm:lsqr}
Let the matrix $A\in \mathbb{R}^{m\times n}$ have full column rank and thin QR decomposition $A = Q_1R$. Then the least squares problem $\min_{x} \norm{b - Ax}$ has the unique solution
\begin{align*}
    x = R^{-1}Q_1^Tb.
\end{align*}
\end{thm}
\begin{proof}
Introducing the QR decomposition of $A$ in the residual vector, we get
\begin{align*}
    \norm{r}^2 = \norm{b - Ax}^2 = \norm{b - Q\begin{bmatrix}R \\ 0\end{bmatrix}x}^2 = \norm{Q^Tb-Q^TQ\begin{bmatrix}R \\ 0\end{bmatrix}x)}^2 = \norm{Q^Tb - \begin{bmatrix}R \\ 0\end{bmatrix}x}^2
\end{align*}
Then we partition $Q = (Q_1\ Q_2)$, where $Q_1\in \mathbb{R}^{m\times n}$, so we can write
\begin{equation}
\label{eq:lsqr}
    \norm{r}^2 = \norm{\begin{bmatrix}Q_1^Tb \\ Q_2^Tb\end{bmatrix} - \begin{bmatrix}Rx \\ 0\end{bmatrix}}^2 = \norm{\begin{bmatrix}Q_1^Tb - Rx \\ Q_2^Tb\end{bmatrix}}^2
\end{equation}
Under the assumption that the columns of $A$ are linearly independent and since the bottom block of equation (\ref{eq:lsqr}) is independent from vector $x$, we can solve $Rx = Q_1^Tb$ and minimize $\norm{r}^2$ by making the upper block in \hyperref[eq:lsqr]{equation (\ref{eq:lsqr})} equal to zero.
\end{proof}
As shown in \parencite[Lecture 10]{Bau} and \parencite[Chap. 5.3]{elden}, in the solution of (\ref{eq:lsqr}) there is no need in computing $Q$ because we only need the product $Q_1^Tb$ to solve the reduced problem $Rx=Q_1^Tb$ as described by \hyperref[thm:lsqr]{\textbf{Theorem \ref{thm:lsqr}}}. We can exploit this fact and apply the following two algorithms to obtain the upper triangular matrix $R$ (\hyperref[alg:hhqr]{Algorithm \ref{alg:hhqr}}) and the product $Q_1^Tb$ (\hyperref[alg:q1b]{Algorithm \ref{alg:q1b}}).

\begin{algorithm}[H]
	\caption{Householder QR factorization}
	\label{alg:hhqr}
	\begin{algorithmic}[1]
		\For{$k = 1, 2, \ldots, min(m,n)$}
		\State $[\mathbf{u}, \mathbf{s}] \leftarrow householder\_vector(\mathbf{A_{k:end, k}})$
		\State $\mathbf{A_{j,j}} \leftarrow \mathbf{s}$
		\State $\mathbf{A_{j+1:end,j}} \leftarrow 0$
		\State $\mathbf{A_{j:end,j+1:end}} \leftarrow \mathbf{A_{j:end,j+1:end}}-2u(u^T(\mathbf{A_{j:end,j+1:end}}))$
		\EndFor
	\end{algorithmic}
\end{algorithm}


\begin{algorithm}[H]
	\caption{Implicit computation of $Q_1^Tb$}
	\label{alg:q1b}
	\begin{algorithmic}[1]
		\For{$k = 1, 2, \ldots, n$}
		\State $\mathbf{b_{k:m}} \leftarrow \mathbf{b_{k:m}} - 2\mathbf{v_k}(\mathbf{v_k^T}\mathbf{b_{k:m}})$
		\EndFor
	\end{algorithmic}
\end{algorithm}
As a final step we give the total amount of operations needed (for a \textit{thin QR factorization}) to solve the \textit{Least Square problem} via QR factorization using Householder transformations. As shown in \parencite[Chap. 5.4]{elden}, we can estimate the number of flops for computing R approximately with:
\begin{align*}
    4\sum_{k=0}^{n-1}(m-k)(n-k) \approx 2mn^2 - \frac{2n^3}{3} + \mathcal{O}(mn),
\end{align*}
and we can state the behaviour in two common regimes:
\begin{itemize}
    \item Square matrices ($\mathbf{m = n}$): $\frac{4}{3}n^3$
    \item $\mathbf{m \gg n}$: scales like $2mn^2$
\end{itemize}